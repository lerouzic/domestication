\documentclass[12pt]{lettre}

\usepackage[utf8]{inputenc}
\usepackage[T1]{fontenc}
%~ \usepackage[french]{babel}

\usepackage{graphicx}
\setlength\openingspace{1.5cm}
\setlength\sigspace{0.2cm}

\setlength{\parskip}{2em}

\begin{document}

\institut{egce-eng}

\begin{letter}{~}

\nodate
\nolieu

\def\concname{Subject:~}
\conc{Submission to Genetics}

\opening{Dear editors, }


We would like to submit the attached manuscript, entitled \emph{Towards the definition of a molecular domestication syndrome: simulations reveal a rewiring of genetic architectures}, for consideration in Genetics. 

This collaborative work by Ewen Burban, Maud Tenaillon, and myself, develops theoretical insights about the expected evolution of the gene regulatory networks during domestication. We suggest that domestication in general should be associated with a molecular syndrome featuring e.g. maladaptive gene expression plasticity, and the rewiring of gene networks leading to a decrease in modularity. These claims are supported by numerical simulations. 

Domestication has always been tightly connected with evolutionary biology, as it can be used as large-scale adaptation experiments featuring directional selection, environmental change, and demographic events.  We thus deem it likely that this work will be of interest for a broad range of population geneticists and molecular biologists, especially those working on domesticated species. 

We suggest that this manuscript could be handled by Dr Stephen Wright as a senior editor, and Dr Graham Coop as an associate editor. Reviewer suggestions were provided in the web submission form. We do not wish to exclude any particular editor or reviewer. 

As indicated in the manuscript, all scripts and programs necessary to reproduce the results and the figures are provided in public gitHub directories. We have no conflict of interest to declare. All co-authors have read and validated the submitted version. 

\closing{On behalf of all co-authors, yours sincerely,}

\end{letter}

\end{document}
